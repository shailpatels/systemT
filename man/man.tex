\documentclass{article}
\usepackage[english]{babel}
\usepackage[utf8]{inputenc}

\usepackage{fancyhdr}
\pagestyle{fancy}
\fancyhf{}
\rhead{}
\lhead{}


\begin{document}
    \section{Operation Manual}
    Quick Reference:\\~\\
    \texttt{ WRT <PTN> }\\
    \texttt{ MOV [L|R] }\\
    \texttt{ RED <PTN> WRT <MRK> }\\
    \texttt{ RED <PTN> MOV [L|R] }\\~\\
    \texttt{ STP }\\
    \texttt{ RED <PTN> STP }\\~\\
    \texttt{ <PTN>: }\\
    \texttt{ JMP <PTN>}\\
    \texttt{ RED <PTN> JMP <MRK>}

    \section{Introduction}
    \texttt{systemT} is an advanced computing machine for general problem solving. 
    Its simple instruction set allows for low compiler overhead \& efficient execution while maintaining clear, readable code.  \\
    \texttt{systemT} is unmatched to any other competitor with its revolutionary 
    unbounded tape - allowing for theoretically infinite memory.

    \section{First Steps}
    \texttt{systemT} revolves around reading and writing to a tape. \\
    The following program will write the string, ``{\it foo}" to the tape.\\ 
    \indent \texttt{WRT foo}\\

    The current cell will now be written over, successive writes will overwrite the
    cell. In order to write to another the tape will need to be moved either left 
    or right in order to point to a new cell.\\
    You can move the tape either left or right\\
    \indent \texttt{MOV L}\\
    \indent \texttt{MOV R}\\

    {\bf Control Flow}\\
    You can modify the control flow of your program through reads and jumps.
    The word ``{\it RED}" will read the current cell and compare it to the next word.
    If the words match, the third word will execute.
    E.g:\\
    \indent \texttt{RED foo MOV L}\\

    The code above will read the current cell  and compares it to the string "{\it foo}", if there's a match the next command will execute, in this case: move left.
    Any command, except another read, can be executed.

    {\bf Jumping}
    You can define labels to jump to a line to execute. A label can be any string immediately followed by a colon. Once defined you can use the ``JMP" word followed by the label name.
    E.g\\
    \indent \texttt{foo:\\\indent JMP foo}\\~\\
    {\it Note: A label must be defined before it can be jumped to, i.e. execution must of reached the label definition first}\\

    Conditional jumps can be performed by following a read immediately by a jump:\\
    \indent \texttt{foo:\\\indent RED bar JMP foo}\\


    {\it Note: label definitions can be overwritten, by creating another definition later on in your program}\\~\\

    The ``{\it STP}" word is unique, it will halt execution and check if the tape 
    is in an accept state, it can also be executed after a read.
\end{document}
